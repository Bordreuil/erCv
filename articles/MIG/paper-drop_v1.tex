%%%%%%%%%%%%%%%%%%%%%%%%%%%%%%%%%%%%%%%%%%%%%%%%%%%%%%%%%%%%%%%%%%%%%%%%
%    INSTITUTE OF PHYSICS PUBLISHING                                   %
%                                                                      %
%   `Preparing an article for publication in an Institute of Physics   %
%    Publishing journal using LaTeX'                                   %
%                                                                      %
%    LaTeX source code `ioplau2e.tex' used to generate `author         %
%    guidelines', the documentation explaining and demonstrating use   %
%    of the Institute of Physics Publishing LaTeX preprint files       %
%    `iopart.cls, iopart12.clo and iopart10.clo'.                      %
%                                                                      %
%    `ioplau2e.tex' itself uses LaTeX with `iopart.cls'                %
%                                                                      %
%%%%%%%%%%%%%%%%%%%%%%%%%%%%%%%%%%
%
%
% First we have a character check
%
% ! exclamation mark    " double quote  
% # hash                ` opening quote (grave)
% & ampersand           ' closing quote (acute)
% $ dollar              % percent       
% ( open parenthesis    ) close paren.  
% - hyphen              = equals sign
% | vertical bar        ~ tilde         
% @ at sign             _ underscore
% { open curly brace    } close curly   
% [ open square         ] close square bracket
% + plus sign           ; semi-colon    
% * asterisk            : colon
% < open angle bracket  > close angle   
% , comma               . full stop
% ? question mark       / forward slash 
% \ backslash           ^ circumflex
%
% ABCDEFGHIJKLMNOPQRSTUVWXYZ 
% abcdefghijklmnopqrstuvwxyz 
% 1234567890
%
%%%%%%%%%%%%%%%%%%%%%%%%%%%%%%%%%%%%%%%%%%%%%%%%%%%%%%%%%%%%%%%%%%%
%
\documentclass[12pt]{iopart}
\usepackage{graphicx}
\usepackage[sf]{subfigure}
%\newcommand{\gguide}{{\it Preparing graphics for IOP journals}}
%Uncomment next line if AMS fonts required
%\usepackage{iopams}  

\begin{document}
\bibliographystyle{unsrt}

\title[Macro drop and droplet profile detection during PGMAW process by image processing]
{Macro drop and droplet profile detection during PGMAW process by image processing}

\author{E Romero, J Chapuis ,C Bordreuil, F Souli\'e, G Fras}

\address{Laboratoire de M\'ecanique et G\'enie Civil,
CC048, Place Eug\`ene Bataillon, Universit\'e Montpellier 2,
34095 Montpellier, France}
\ead{cyril.bordreuil@univ-montp2.fr}

\begin{abstract}
The paper describes some new image treatment algorithm used to 
detect profiles during arc welding process. The new algorithm 
are an agregation of some available algorithm of image treatement,
computational geometry and graph theory. The algorithm allows to extract precise
geometrical entities as closed or open profiles that could be used for monitoring
of welding process. The algorithm even if used external libraries is really efficient
and could be used for real time monitoring.
\end{abstract}

%Uncomment for PACS numbers title message
%\pacs{00.00, 20.00, 42.10}
% Keywords required only for MST, PB, PMB, PM, JOA, JOB? 
%\vspace{2pc}
\noindent{\it Keywords}: Arc welding, Image treatment, Melt pool,  Profile detection, Monitoring, 
% Uncomment for Submitted to journal title message
\submitto{\MST}
% Comment out if separate title page not required
\maketitle

\section{Introduction}

A better observation of behavior of the weld pool or the metal droplet in a Pulsed Gas Metal Arc Welding (PGMAW)
 process could help to enhance welding quality in manufacturing process \cite{LIN}, \cite{WU1}.
 In a PGMAW process, the weld quality is strongly related with the metal transfer  stability 
 (droplet deposition). The monitoring of metal transfer process
 will indicate the good deposition process \cite{WANG}. 
The macro drop kinetic measurement during static PGMAW process could improve the
understanding of some GMAW related phenomenon  \cite{CHO} as free surface 
fluctuations \cite{WANG}. The measurement of the geometry of the free surface 
of macrodrop or droplets is an important information for the understanding of the physical phenomenom 
during welding and
the validation of the simulation of welding process.

In this article, the monitoring of the shape and size of these welding objects during
static PGMAW  is 
investigated with a 2D method. A shadowgraphy technique, or back lighting,
is the natural choice to record the droplets and macro drops profiles \cite{BALSAMO}.
 Due to the arc light interference and the relatively high 
speed of wire feed process, a high speed camera and an effective image processing algorithms are 
required \cite{WANG}. The  frame rate acquisition to metal transfer drop kinetic analysis is $3000$ 
per second. Therefore the algorithms have to be able to extract 
the geometrical information (area, size and others) from the macro drop and droplets,
from a huge amount of data. In addition, the voltage and current signals are directly related 
with the droplet formation at the wire \cite{BALSAMO}.
 

There have been many studies on visual sensing techniques for observing weld pool 
image \cite{BAE} and metal transfer process during welding \cite{LIN}.
Optical sensors like high speed  cameras and lighting systems have been
widely use in GMAW process to realize  image acquisition \cite{ZHANG4},
control process \cite{BAE}, parametric studies \cite{BALSAMO} and droplet dynamics analysis\cite{LIN}.
Image processing plays a critical role in extracting useful information from visual scenes \cite{WANG}.
Nevertheless the strong interference from the arc lightning required more than standard 
image treatment to analyse the raw images of the welding process \cite{NORDBRUCH}. 
Previous work has shown that is possible to perform geometrical 
analysis in weld pools or droplets \cite{WU1}.
Parameters such as macro drop or droplets surface, volumes 
or height has been measured using different and 
specific processing images algorithms \cite{WANG}, \cite{WU1}, \cite{SEED},
\cite{NORDBRUCH}. However, to date, effective automatic image processing
of metal transfer has not been developed, possibly due to the  difficulty involved 
for welding researchers \cite{WANG}. 

To perform geometrical analysis in macrodrop and droplet,
 a multipurpose C++  library (erCv) was developed at the lab. 
These library results from selected functions  from highly reliable 
open source libraries to images treatment, geometrical 
analysis, graph theory applications and image visualization.
In a first step, and despite the different static and dynamic weld conditions, 
as well as different current regime. A reliable 2D profiles and geometrical
parameters from macro drop and droplets, has been obtained using the mentioned library.
  The purpose is to help simplification of future numerical models and enhance monitoring by direct observation.
     
The paper is organized as follows. First, the experimental setup that will allow to appreciate the performance of the algorithm,
 to study macro drop analysis is detailed in order to appreciate the difficulies of profile detection
during welding process. 
Then The library and the algorithms are explained and some results are shown on the macro drop and droplet study.
The objective  is to perform geometrical analysis and measure characteristic
times of the macro drop growth and the droplet during the free flight 
during  a static PGMAW process. 


\section{Experimental Setup}
\label{experimental_setup}

\subsection{Multi-physics platform}
\label{multi_physics_platform}


A specific data acquisition setup was done at the lab in order to
synchronize all kind of signals and images versus time and space \cite{ChapuisThesis}.
The experimental setup is composed of a welding generator, a XY table, a data acquisition
system and a high speed camera as shown on figure \ref{schema-montage-experimental-GMAW}.
In this article only details on image acquisition setup and treatment will be outlined.



\subsection{ Image acquisition setup}
\label{ image_acquisition_setup}

The PGMAW static process is recorded by Shadowgraphy optical method. 
An halogen lamp is used to light the weld process. To guarantee an homogeneous
illumination of the welding process, a light diffuser is placed
in the optical path closer to the halogen lamp. 
Finally the shadows of welding elements are projected to the other side where
 a Phantom V5.0 high speed camera is placed and aligned in
 the optical path (see figure \ref{schema-montage-experimental-GMAW}). 

\begin{figure}
\begin{center}
\includegraphics[width=7cm,height=4cm]{images/setup.png}
\caption{{\small Experimental setup to detect macro drop and droplet edges in GMAW process}}
\label{schema-montage-experimental-GMAW}
\end{center}
\end{figure}

To enhance the image contrast of the weld elements inside the electric discharge,
the intensity rate between arc light and halogen lamp have to be reduced. In order to this
a $650\ \pm 10\ nm$ band pass filter is placed in front the camera lens. 650nm
is around a no emitting wavelength from the arc. Nevertheless, the raw images remain highly 
noisy by the arc light. 


\subsection{ Welding condition}
\label{ welding_conditions}

Stationary spots weld are made using the GMAW process with the Oerlikon CitoWave 500 generator. 
The target is a steel disk of $10\ mm$ of thickness and 80mm radius.
The fill wire is ER70S steel welding of 1mm diameter.
The  wire feed speed is set to $6\ m/min$. The frequency droplets is set to $113\ Hz$. 
Two kind of  shielding gases were used \ref{tab::gases}. 
\begin{table}[h]
\centering
\begin{tabular}{|c|c|c|}
\hline
Gaz & Argon (\%) & $CO_2$(\%) \\ \hline
G1 & 100       & 0 \\ \hline
G2 & 92        & 8 \\ \hline
\end{tabular}
\caption{Shielding gas composition}\label{tab::gases}
\end{table}

Some welding parameters values are summarized in tables \ref{table-parameters-static}.
For the experiments described all along the paper there is no welding speed.

\begin{table}
\begin{center}
\begin{tabular}{|cc|}
\hline
Welding wire type & ER70S \\ 
Wire diameter ($mm$) or $drw$ & 1 \\  
Contact tip to work distance ($mm$) & 20 \\
Shielding gaz flowrate ($l/min$) & 18 \\ \hline
\end{tabular}
\caption{{\small Determined constant welding parameters used in experiments}}
\label{table-parameters-static}
\end{center}
\end{table}



%Welding current and arc voltage are recorded at $30\ kHz$ sampling rate.
The images are recorded at $4000$ frames per seconds, which is enough to
measure macro drop radius and apparent liquid-solid contact angle histories. 
%The images and electrical signals are synchronize, thanks to the
%automatically approach made in the multi physics platform.

%Measurements are made using the erCv library.
      
      
      
      
\subsection{ System definitions}
\label{system_definitions}
\subsubsection{ Macro Drop}
\label{ macro_drop}

%The purpose is to study the evolution of weld pool object in
%a GMAW process or macro drop. Therefore it is necessary to study the shape and spreading
%of the macro drop according depositing droplets of feed wire. 

At figure \ref{schema-macro-drop-droplet-parameters} appears the geometrical 
elements to be study at the macro drop: the macro drop radius at the base $R_{m}$, 
the macro drop center height $h_{m}$, the macro drop volume $V_{m}$, the
penetration below the surface level $p$ and the wetting angles:
$\Theta_{m(al)}$, $\Theta_{m(ar)}$ (apparent) and $\Theta_{m(rl)}$, $\Theta_{m(rr)}$ (real).
With the indices $l$ and $r$ refer to left and
right angles. The indices: $m$ refer to macro drop, $g$ refer to gaz, $l$ refer
to liquid, $s$ refer to solid$/$substrate and $a$ refer to the droplet.
The angles and the base are the two main quantities to be measured. but the overall shape of
the macrodrop is an important information for computation.

\begin{figure}
\begin{center}
\includegraphics[width=7cm,height=4cm]{images/schema-macro-drop-droplet-parameters.png}
\caption{{\small Profile schematics of welding objects in a GMAW process}}
\label{schema-macro-drop-droplet-parameters}
\end{center}
\end{figure}
    
\subsubsection{ Metal Transfer Drop}
\label{ metam_transfer_drop}

For the droplet or metal transfer drop, the geometrical parameters to identify are:
The estimate volume $V_{a}$ and the estimate radius $R_{a}$. 
Finally in order to better understand the droplet dynamic \cite{}, 
the physical elements to identify are: the fall time $t_{a}$ and the average 
speed $U_{a}$ (see figure \ref{schema-macro-drop-droplet-parameters}).



\section{Image processing} 

Some definitions and brief description of general principles used in
 image processing are required to measure the profiles. Then, the design of the library is briefly detailed
to succeed to detect the profile in the noisy environment.
\subsection{Image treatment basics}
\subsubsection{Some definitions}\label{some_definitions}
A numerical grey image can be describe as 3D surface discretized by a grid mesh
 in the X,Y plane. Each mesh represents a pixel and the relief surface at Z axis,
 the grey level. 
A strong relief change or high gradient greyness values at
 the image are perceive by human eyes as light changes, and can be interpreted
 as objects edges. 

Sometimes, a regular relief patrons or regular greyness
 variation can be distinguish. The human eyes can perceive these patrons as texture 
and interpret the space between different textures zones as edges.
There exist a large spectrum of algorithms to image treatment, in particular to edges detect.
 Most of them can be classified by the way that his operate above the image pixels and grey level.

\subsubsection{Filters}\label{filters}

The filters are algorithms that operate as mathematical functions $f$ above the 
$X$, $Y$ or both axis of the image ($Z = f(X,Y)$,  
$f(X)$ or $f(Y)$)  modifying his greyness value or $Z$ component. 
Different kind of filters can be mentioned as median, Gaussian, impulse,
 adaptive and others. The impulse filters such Canny are 
widely use to edge detection \cite{COCQUEREZ}. These filter have an
 impulse response to most important greyness gradient in the 
image; this allows the filter a better edges localisation
 (see figure \ref{photo-explication-filter}). 
For this reason, Canny filter is widely 
used in the library to detect the welding elements edges. 
However, it is sensible to noise or secondary greyness gradients in the image
 and, in consequence, it have some difficulties to define closer surface.

\subsubsection{Snake and level set}
\label{snake-and-level-set}
 
Curve propagation is a popular technique in image analysis for object extraction,
 object tracking, edge detection and others (see figure 
\ref{photo-explication-snake}). The central idea behind such an approach
 is to evolve a curve towards the lowest potential of a cost function. 
However at each stage of curve evolution, each curve point potential has
 to be computed. A lot of point (better curve resolution) take a lot 
computing time, and therefore are not yet apply to real time detection
 or relatively speed automatic image processing. For this reason snake algorithms are not used in the library.

\subsubsection{Segmentation}
\label{segmentation}

Let $B$ an image and let $R_{i}$ a region of $B$ such:

\begin{eqnarray}
B = \bigcup_{i}R_{i}\ \forall i \in \{0, \mbox{numbers of regions in B}\} \\
\mbox{with} R_{i} \neq \emptyset \\
\mbox{and}  R_{i}\bigcap R_{j} = \emptyset  \forall i, j\ \mbox{with}\ i \neq j\ 
\label{equation-segmentation}
\end{eqnarray}

A $B$ segmentation is an image treatment which generate a $B$ partition in $R_{i}$ regions. 
Each region is a connected set of pixels with 
common properties (intensity, texture,...) \cite{COCQUEREZ}.
 The partition is generated by operations or comparisons methods between 
regions. Generally, this treatment offers a good detection edges if
 the elements and surrounding area have different textures (see figure
 \ref{photo-explication-segmentation}). 
Note that different regions can belong to the same partition and not
 be placed together, therefore the surface is not always connected and,
 in consequence, the edges of the interest regions are not always closed.

\begin{figure}[h!]
\begin{center}    
\subfigure[Weld pool image in static GTAW process]{\label{photo-explication-patron}\includegraphics[width=3.5cm,height=3.5cm]{images/photo-explication-patron.png}}
\subfigure[Canny filter treatment]{\label{photo-explication-filter}\includegraphics[width=3.5cm,height=3.5cm]{images/photo-explication-filter.png}}\\
\subfigure[Snake treatment]{\label{photo-explication-snake}\includegraphics[width=3.5cm,height=3.5cm]{images/photo-explication-snake.png}}
\subfigure[Segmentation image by 2 cluster sample comparison]{\label{photo-explication-segmentation}\includegraphics[width=3.5cm,height=3.5cm]{images/photo-explication-segmentation.png}}
\end{center}
\caption{{\small Samples of different methods technique for image processing}}
\label{photo-explication}
\end{figure}


\subsection{Image treatment library}
\label{image-treatment-library-ercv}

As mentioned, a multipurpose image processing library was developed 
and currently use at the laboratory, in order to analyze welding process objects.
erCv is able to perform edge detection and geometrical analysis in a different 
welding objects such as Macro drop, droplets and weld pool.
erCv is a modular library assembled in a oriented object C++ language. Then erCv is a scalable 
and portable library able to perform real time
 contour detection. The library is implemented in  C++ with some bindings in python,
 making it relatively convivial to use for non-programmers.
The library is composed by four processing modules: 
\begin{description}
\item[Image Treatment:] Due to weld process conditions such arc lightening, 
 heat and electrodes positions; the raw image registered by CCD 
 camera are not calibrated and present light inhomogeneities and 
 noise. In order to obtain the real shape and size of weld elements, this module 
 includes calibration algorithms. To detect the welding objects contours it is 
 necessary to improve the weld element image. This module has the 
 pre-processing treatments to noise reduction and image enhancement.
 Then to start the edges detection process, this module includes processing
 algorithms as segmentations by samples comparators, watershed transformation,
 filters edge detectors and histogram based methods.  
\item[Geometrical Treatment and Analysis:] This module convert edges pixels points into connected 
   segments to conclude the edge detection
  process, completing and in some cases extrapolating the weld
  elements edge. It is also responsible to compute the geometrical 
  data of welding elements such weld pool surface and metal transfer drop volumes. 
  This module uses a full geometry algorithm library\cite{CGAL}, which include different algorithms
  such as triangulations and mesh generation, alpha shape and 
  convex hull generation and polygonal structures.
\item[Graph Theories:] To compute the geometrical data of welding elements
   it is necessary to extract the welding object edge from the image; this
   required some criteria such as continuity, length or closer condition.
   This module use graph algorithms to  identify and select the welding element edge using the
   criteria. This module is composed by connected segments, estimates
   minimal cut, determine largest chain segments and others algorithms. 
\item[Visualization:] This module is a set of 
  functions use to execute, show and/or register the 
  different steps at the image process.
\end{description}

\begin{figure}
\begin{center}
\includegraphics[width=10cm]{images/schema-erCv.png}
\caption{{\small Flow diagram of erCv library composition}}
\label{schema-erCv}
\end{center}
\end{figure}

The originality of the library comes from the agregation 
and complementarity of the different numerical tools.
The main difficulty is to find algorithms with good parameters all along the process.

\subsection{ The Analyzing methods}
\label{ the-analyze-methods}
\subsubsection{ Image Calibration}

As was shown in section \ref{image-acquisition-setup}, the camera recording a 
shadow projection of the welding process. The projection is 
supposed to be a section of the welding process image, orthogonal to the optical path.
Therefore, to a first approach, not image projection 
correction is necessary. Only to calibrate the image dimension 
(at pixels) to the real objects dimension (at millimeters), a conversion scale is 
made. In our experiment, the known width of the welding wire leads to $conv = 1\ mm/ 20\ pixels$ . 
  
  
\subsubsection{ Image Treatment}
\label{ image_treatment}

The idea is to extract the macro drop and droplet shapes 
from the raw images and therefore to extract the geometrical information.
 The most common approach is to segment the image \cite{WANG}. 
That means to separate in two fields (color) the image. The macro drop and droplet in one color and
the rest of the image in other \cite{COCQUEREZ}. Thanks to the optical method acquisition
 (shadowgraphy) the images are already segmented between
light and shadow zones (see figures \ref{photo-macro-drop-original});
 where the shadow zones are the projection of the macro drop and droplet.

However objects within the image under the arc may differ only by few intensity light level
from the background. In these images areas, a contrast enhancement is necessary. 
Objects outside the arc, as  some regions of the macro drop profile,
 are clearly visible against the background. In these regions a contrast enhancement 
can lead the image quality and thus an incorrect segmentation \cite{NORDBRUCH}.
 Therefore the contrast enhancements, or threshold, have to be adapted per regions in the image.  

To simplify the contrast enhancement, the images are converted from RGB image color 
to grey levels image (see figure \ref{photo-macro-drop-grey}). To 
reduce the grey level noise, a first smooth classical filter 
(Blur) is applied over the image. The filter replace
the grey values of squares of $7\times 7$ 
pixels by they average value (see figure \ref{photo-macro-drop-smoothblur}). 
In consequence, the homogeneity quality is improved into the light and 
shadows regions, despite the edge contrast reduction induce between these regions. 

Then, a  adaptive threshold can be used to enhance the difference
between shadow elements and light zones. To the adaptive threshold it is important 
to choose the correct size region where it will be applied. Small regions to threshold
will generate isolated patterns (groups of pixels). Big regions will generate
a classical like threshold image \cite{SABER}. The region size dominium choose
to threshold application is $23\times 23\ pixels$. The threshold criteria 
choose is binary: $greyL_{thres}(x,y) = max_{value}$ if $greyL_{origin}(x,y) >  threshold(x,y)$
or $greyL_{thres}(x,y) = 0$ otherwise, $\forall (x,y)$ inside the region
to threshold \cite{OPENCV}. Where $threshold(x,y)$ is the threshold parameter,
 $max(value)$ is the grey level value choose to identify the pixels above
threshold parameter, $greyL_{origin}(x,y)$ is the grey level value of pixel
 before threshold and $greyL_{thres}(x,y)$ is the pixel grey level after threshold.
$threshold(x,y)$ is obtain by subtraction between the average grey level of region
 pixels neighborhood and a constant parameter $C_{threshold}$.  
Small edge defects can still be found, due to the previous smooth filter and some arc 
light interference regions, which are bigger than regions size to
threshold (see figure \ref{photo-macro-drop-adaptivethreshold}).  

A second filter type median is necessary to improve the previous threshold treatment. 
This filter compute the grayness median value over $7\times 7\ pixels$
region. The image, which is already binary type due to adaptive threshold processing,
 is enhancing in his mixed black white regions depending of 
 median value. In consequence contour quality between light and shadows
 regions at the binary image are improved (see figure \ref{photo-macro-drop-smoothmedian}).

Finally an impulse response filter to grey level gradient (Canny) is applied to extract 
the principal edges at image. In a binary image, all edges are principal,
therefore canny detect all edge at the image (see figure \ref{photo-macro-drop-canny}).
Nevertheless to guarantee a full detection of droplet and 
 macro drop profile, the canny parameters are adjusted to maximum, 
which means maximum edge detection sensibility. 
              
\begin{figure}[h!]
\begin{center}    
\subfigure[Macro drop raw image]{\label{photo-macro-drop-patron}\includegraphics[width=3.5cm,height=3.5cm]{images/photo-macro-drop-original.png}}
\subfigure[Macro drop grayness image]{\label{photo-macro-drop-grey}\includegraphics[width=3.5cm,height=3.5cm]{images/photo-macro-drop-grey.png}}\\
\subfigure[First smooth filter]{\label{photo-macro-drop-smoothblur}\includegraphics[width=3.5cm,height=3.5cm]{images/photo-macro-drop-smoothblur.png}}
\subfigure[Adaptive threshold per regions]{\label{photo-macro-drop-adaptivethreshold}\includegraphics[width=3.5cm,height=3.5cm]{images/photo-macro-drop-adaptivethreshold.png}}\\
\subfigure[Median filter]{\label{photo-macro-drop-smoothmedian}\includegraphics[width=3.5cm,height=3.5cm]{images/photo-macro-drop-smoothmedian.png}}
\subfigure[Canny filter]{\label{photo-macro-drop-canny}\includegraphics[width=3.5cm,height=3.5cm]{images/photo-macro-drop-canny.png}}\\
\end{center}
\caption{{\small Image treatment step by step of PGMAW process}}
\label{photo-macro-drop}
\end{figure}


\subsubsection{Edge extraction}
\label{edge_extraction}
Despite performance achieve by the image treatment, Canny filter
 generate many edges, or cords, in the image. These cord are composed
 by white pixels, interpreted by the algorithms as an individuals points.
 In order to automate the edge extraction process, geometrical and
 graph algorithms are used  to isolate the cords corresponding to
 macro drop profile and edge section droplet.

To the macro drop profile a user interaction algorithm is responsible
 to extract the continuous cord (white pixels) corresponding
 to the profile into a list. Note  that the macro drop and 
the substrate have the same continuous profile (see figure \ref{photo-macro-drop-canny}.
 The substrate shadow projection is always the same if the optical
 path doesn't change. Therefore the users have only to choose, the
 first image in the experience series, one extreme  of the cord 
corresponding to substrate profile (see figures \ref{photo-results-macro-drop}).
 Then the algorithm finds all new white pixels inside small neighbor around
 the pixels selected by the user. Each new white pixel  is added to the
 profile list. Starting at each new pixel, this process continues,
 automatically, until the other extreme of the cord, until the last
 image of the series.
 
\begin{figure}[h!]
\begin{center}    
\subfigure[t = ]{\label{photo-results-macro-drop-t1}\includegraphics[width=3.5cm,height=3.5cm]{images/photo-results-macro-drop-t1.png}}
\subfigure[t = ]{\label{photo-results-macro-drop-t2}\includegraphics[width=3.5cm,height=3.5cm]{images/photo-results-macro-drop-t2.png}}\\
\subfigure[t = ]{\label{photo-results-macro-drop-t3}\includegraphics[width=3.5cm,height=3.5cm]{images/photo-results-macro-drop-t3.png}}
\subfigure[t = ]{\label{photo-results-macro-drop-t4}\includegraphics[width=3.5cm,height=3.5cm]{images/photo-results-macro-drop-t4.png}}\\
\end{center}
\caption{{\small Images series of macro drop profiles}}
\label{fig::photo-results-macro-drop}
\end{figure}
               
               
To the droplet edge section, the white points cord defines 
the surface of the droplet. Note at figure \ref{photo-droplet-canny} that the biggest closer
 cord correspond to the droplet section. The idea is to build segments around
 the cords, and take the longest closer segment. 
                
                
Give $S$ the set of points in the space image. The segments could 
be built between wherever two points of $S$. However, to fulfill
 the cords with segments, only the points placed together could define
 a segment. To choose the correct points of $S$ a Delaunay triangulation
 is made over $S$ \cite{CGAL}. Each point of $S$ is a vertex
of a triangle and then each potential segment is a edge of a triangle. 
The Delaunay triangulations guarantee a unique solution with minimal
 internal angles for each triangle, which means  minimal
 length sides to the triangles \cite{GEOMETRICAL}.
 Give $\beta$ the cord surrounding the external hull of the Delaunay
 triangulation of $S$. An alpha shape algorithm uses a $\alpha \in N$ parameter,
 or square radius of a test circle, to travel from outside $\beta$ to inside
 the triangulation. The circle has to go through the triangles sides without touching the vertices. Give 
$n$ the number of triangles sides of triangulation
and $l_{i}$ with $i\in\{1,n\}$ the length of each side, if $\sqrt{\alpha} > l_{i}$ 
with $i \in\{1,n\}$, a segment is marked between the respective vertices or
 points. At the end of the process, a  set of segments knows as $\alpha$-shape has been created.
  Note if $\sqrt{\alpha} > l_{i} \forall i$, the circle can't go inside $\beta$, then $\alpha$-shape
 is $\beta$ \cite{CGAL}. If $\sqrt{\alpha} < l_{i} \forall i$ the circle go through
 every side in the triangulation,  then no segment is created and $\alpha$-shape
 became $S$. The points that compound the droplet section are all contact neighbors. Then $\alpha = 1$
 is sufficient to create a segment set around the droplet
 (see figure \ref{photo-droplet-alpha-shape}). 
                                  
\begin{figure}[h!]
\begin{center}    
\subfigure[t = ]{\label{photo-droplet-canny}\includegraphics[width=3.5cm,height=3.5cm]{images/photo-droplet-canny.png}}
\subfigure[t = ]{\label{photo-droplet-alpha-shape}\includegraphics[width=3.5cm,height=3.5cm]{images/photo-droplet-alpha-shape.png}}\\
\end{center}
\caption{{\small Images series of macro drop profiles}}
\label{photo-results-droplet}
\end{figure}

A graph algorithm connects this segment and finds the longest closer segment.
 The longest closer segment corresponds to the metal transfer drop profile 
(see figure \ref{photo-results-droplet}).

\begin{figure}[h!]
\begin{center}    
\subfigure[t = ]{\label{photo-results-droplet-t1}\includegraphics[width=3.5cm,height=3.5cm]{images/photo-results-droplet-t1.png}}
\subfigure[t = ]{\label{photo-results-droplet-t2}\includegraphics[width=3.5cm,height=3.5cm]{images/photo-results-droplet-t2.png}}\\
\subfigure[t = ]{\label{photo-results-droplet-t3}\includegraphics[width=3.5cm,height=3.5cm]{images/photo-results-droplet-t3.png}}
\subfigure[t = ]{\label{photo-results-droplet-t4}\includegraphics[width=3.5cm,height=3.5cm]{images/photo-results-droplet-t4.png}}\\
\end{center}
\caption{{\small Images series of macro drop profiles}}
\label{photo-results-droplet}
\end{figure}


\subsection{Geometrical Analysis}
\label{geometrical_analize}

Now it is possible to determine the geometrical parameters shown 
at figure \ref{schema-macro-drop-droplet-parameters}. To compute the wetting 
angle measure, linear regressions or least squares (parabolic, ellipse, circle) 
are taken from the processed macro drop extracted profiles.
A loop algorithm applies this method to all frames at the experience to compute the wetting 
angles. Same procedure is used to estimate the macro drop radius and volume \cite{CHAPUIS}.

The metal transfert droplet is approximate and close with a combination of $\alpha$-shape and
graph algorithm. Once the set of segments are extracted a triangulation is created. The triangles 
are then used to compute the surface and the position of the centroid. It is assumed that there is a kind
of axisymmetry  and an homogeneous density in the droplet. With the triangles, the principal axis could also be 
computed.



\section{ Results}
In this part some results, performance and reliability of the library for a set of experiments
are shown. The basic experiment is a static PGMAW.
A macrodrop is feed by droplets that are assumed to flight vertically and
then feed the macrodrop. The macrodrop grows with a geometry that depends on the effects of gravity,
surface tension, heat transfert,...

To improve weld pool stability in pulsed GMAW, the growth of the droplet is investigated.
To improve weld quality, the transfert during some pulses are analyzed.

The algoritms are tested on a i7 processor from Intel with 1.87Ghz RAM.


\subsection{ Macro Drop growth investigation}
The investigation of the macro drop growth for the two shielding gas needs the study of two quantities:

\begin{itemize}
\item The speed of growth of the macro drop radius
\item The angle between the substrate and the liquid
\end{itemize}

The first is an indicator of the ease of growth of the macrodrop due to
heat transfert and tension surface and the second  item gives information
 on the capillarity mechanism involved in the macrodrop


The shielding gas has a great effect between the liquid and the gas interface. Gas could diffuse in the
first layer of the liquid and greatly modify the physical properties. The tension surface is the main physical
property that will be modified. A difference of tension surface will modify the shape of the interface by modifying
its equilibrium.
% It has to be kept in mind that modification of shielding composition will also modify
% the plasma energetical and momentum distribution that could also  influence the shape of the interface.

 To study the influence of the gases on the shape of the gas liquid interface, images must be selected in cold
regime of the plasma in the pulse. It is better to select images at the end of the period when the macrodrop is
stabilized. Once the images are selected, the algorithm described in section \ref{image_treatment} is used.
The treatment of each image lasts around 5ms.
\subsubsection{Basis and angle evolution} 
Some profile detection are shown on figure \ref{fig::photo-results-macro-drop} on the macrodrop.
The shape is shown on figure \ref{fig::photo-results-macro-drop}.

 The developped algorithm manage to detect profile under the arc.
 To isolate only 
macrodrop profile, a thershold relative to the base material is used to extract the macrodrop profile.
The result is shown on figure \ref{fig:profile_extract}. Then, basis and angle of the profile can be computed
all along  the process.
\begin{figure}[h!]
\centering    
\includegraphics[width=10cm]{images/CO2_2s.png}
\caption{Profile of macro drop for G1 gas after 2s of welding}
\label{fig::profile_extract}
\end{figure}
First, the evolution of the basis is extracted on figure \ref{fig::basis} by finding 
the min of the max of the contour on figure \ref{fig::profile_extract}.

\begin{figure}[h!]
\centering    
\includegraphics[width=7cm,angle=90]{images/Evolution_Base_Gazs.pdf}
\caption{Macrodrop growth (basis) in function of welding time}
\label{fig::macro_drop}
\end{figure}

The figure \ref{fig::basis} shows the growth of the macrodrop with the two different gas.
The evolution is almost the same in value. The 8 per cent CO2 gaz seems to be a little bit 
more noisy than the pure Argon one. 

Another important geometrical entity is the angle of the macrodrop surface on the substrate.
 To approximate it, a simple linear regression is
performed for the first and last points of the profile. The result are shown on figure \ref{fig::angles}
for the left angle and the two gases.
\begin{figure}[h!]
\centering    
\includegraphics[width=7cm,angle=90]{images/Evolution_Angle_Gazs.pdf}
\caption{Macrodrop left Angle in function of welding time}
\label{fig::angles}
\end{figure}

The evolution curves are also noisy. Videos investigation shows that the macrodrop oscillates during welding. 
Macrodrop oscillations induced  wetting angle oscillations. Depending of the main oscillations during
a period, the local angle oscillation is more or less important. For pure Argon
gas, the equilibrium angle (60 degrees) is rapidly reached and stays constant all along the welding.
For the 8 per cent $CO_2$ gas, the angle increase until reaching a constant value of 60 degrees.

\subsubsection{Discussions}
The noise in the measure for the basis could be attributed
to oscillations of the macrodrop coupled to the threshold for the detection of the macrodrop.

The interface between the gas and liquid is extremely strained. During a pulse, the surface is perturbated by the
increase of pressure in the arc due to increase of current and by the impingement of the droplet inside the
weld pool. These perturbations create oscillations of the macrodrop and on the free surface.
 These phenomenon induce also oscillations near the triple joint.

 The growth of the macrodrop is a balance mainly between
the energy balance (thermo convection) and the mechanical equilibrium (Dupr\'e's law and membrane equilibrium). 
The shape of the macrodrop is an important information for
the local and global equilibrium of the interface. 
For example, the macrodrop made with  pure Argon grows by step denoting an interesting instability 
of the interface. 

The modification of gas influences surface tension but also modifies the energy transfert from the plasma
to the workpiece. Even with same welding parameters, 
the difference of gas will modify the current lines distribution but also the arc length.
Nevertheless, the evolution of the basis and the  shape 
(figure \ref{fig::comp_gazs}) for the two gases are almost the same (figure \ref{fig::macrodrop}).
 This means that at first order the two evolutions
are governed with the same mechanism (mass balance) and then the difference is due to heat transfert and surface tension.
Pure Argon shielding leads to a higher desequilibrium at the interface leading to a higher surface tension. Then the angle
at the triple joint (liquid-gas-substrate) is higher modifying then the behaviour of the overall interface.

\begin{figure}[h!]
\centering    
\includegraphics[width=7cm,angle=90]{images/Comparaison_gazs_2s.pdf}
\caption{Macrodrop shape at 2s for the two gases}
\label{fig::comp_angles}
\end{figure}

This investigation demonstrates the interest of these kind of analysis for understanding of the phenomenon
involved in welding. It is also precise information for validation or comparison with numerical
computation.

\subsection{ Droplet flight}
Droplet flight investigation is of interest for weld quality and the metal deposition process.
The determination of the motion and the impact point is then a good indicator of the quality of the
process. The droplet flight is only investigated for the 8 per cent $CO_2$ for a pulse of 113 Hertz.
The treatment of each image takes between 5 to 45ms with geometrical algorithm (triangulation) and output
(surface computation and position of the center of mass), within the python interpreter.

\subsubsection{Trajectories and droplet velocities}
During one pulse, the droplet are detected. The shape are shown on figure \ref{fig::droplets}.

\begin{figure}[h!]
\centering    
\includegraphics[width=7cm,angle=90]{images/FreeFlight.pdf}
\caption{One droplet at two different times during the free flight}
\label{fig::droplets}
\end{figure}

The shape of the droplet changes during the free flight denoting oscillations of the droplet.
Once the points of the contour of the droplet are detected, the center of mass are extracted  as well as the principal axis.
The center of mass is drawn during several pulses on figure \ref{fig::center_mass}.

\begin{figure}[h!]
\centering    
\includegraphics[width=7cm,angle=90]{images/Trajectory.pdf}
\caption{Locations of the center of mass of the droplet for four different pulses}
\label{fig::center_mass}
\end{figure}
The motions of the center of mass are almost linear and repetitive for the different transferts.
The trajectories show that the motion is aligned more or less with the vertical axis.
The position relative to the vertical axis can then be drawn for the four free 
flight of droplets (figure \ref{fig::velocities}).
 

\begin{figure}[h!]
\centering    
\includegraphics[width=7cm,angle=90]{images/Velocities.pdf}
\caption{Locations of the center of mass of the droplet for four different pulses}
\label{fig::velocities}
\end{figure}

The position of the center of mass of the droplet relative to the vertical axis shows that the 
droplet has a constant speed. The speed is estimated between 150mm/s to 200mm/s depending on the pulse
and the free flight of the droplet last 6ms.

\subsubsection{Discussions}
The oscillations of the droplet gives informations on the competition between surface tension and gravity
\cite{White}.
For the welding conditions of the experiment, the trajectories of the different pulse indicates 
that the forces acting on the droplet are in equilibrium
or are negligible in respect to the size of the droplet.

Despite the two dimensional analysis, the result gives important informations for 
simulation of metal transfert process. If  the droplet is assumed to be a sphere
 the kinetic energy of the droplet just before impact could be estimated. For our case,
The radius of the droplet is estimated at 0.56mm. % and the kinetic energy is around 6e-8J.

Even if the four trajectories are similar, some discrepancies could be outlined. The short time 
for the analysis let nice perspectives for the algorithm for the monitoring of the transfert.


\section{ Conclusions}

%\bibliography{paper_bain}

%\section*{References}
%\begin{thebibliography}{10}
%\bibitem{book1} Goosens M, Rahtz S and Mittelbach F 1997 {\it The \LaTeX\ Graphics Companion\/} 
%(Reading, MA: Addison-Wesley)
%\bibitem{eps} Reckdahl K 1997 {\it Using Imported Graphics in \LaTeX\ } (search CTAN for the file `epslatex.pdf')

% \cite{BALSAMO}
%\cite{WANG}
%\cite{CHO}
%\cite{LIN}
% \cite{ZHANG4}
%\cite{BAE}
%\cite{NORDBRUCH}
%\bibitem{CGAL} http://www.cgal.org 
%\cite{ChapuisThesis}
%\cite{WU1}
%\cite{SEED}
% \cite{COCQUEREZ}
% \cite{OPENCV}
% \cite{GEOMETRICAL}
%\end{thebibliography}
\end{document}

